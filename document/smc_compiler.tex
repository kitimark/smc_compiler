\documentclass{article}
\usepackage[english,thai]{babel}
\usepackage[utf8x]{inputenc}
\usepackage{fonts-tlwg}
\usepackage{tikz}
\usetikzlibrary{shapes,arrows,snakes}



\usepackage{listings}

\lstset{
  basicstyle=\itshape,
  xleftmargin=3em,
  literate={->}{$\rightarrow$}{2}
           {α}{$\alpha$}{1}
           {δ}{$\delta$}{1}
}


\begin{document}

\begin{titlepage}
    \vspace*{\stretch{1.0}}
    \begin{center}
       \Large\textbf{SMC Complier}\\
    \end{center}
    \vspace*{\stretch{2.0}}
 \end{titlepage}

\section{Compiler}
แนวคิดในการทำงานในส่วนของงานทั้งหมด
\begin{center}
\tikzstyle{block} = [rectangle, draw, 
    text width=6em, text centered, rounded corners, minimum height=3em]
\tikzstyle{line} = [draw, -latex']

\begin{tikzpicture}[node distance = 2cm, auto]
    % Place nodes
    \node[draw=none,fill=none] (source) {Source code};
    \node [block, below of=source] (init) {lexer};
    \node [block, below of=init] (parser) {parser};
    \node [block, below of=parser] (interpreter) {interpreter};
    \node[draw=none,fill=none,below of=interpreter] (bi) {binary};
    
    % Draw edges
    \path [line] (source) -- (init);
    \path [line] (init) -- node {token} (parser);
    \path [line] (parser) -- node {parsed tree} (interpreter);
    \path [line] (interpreter) -- (bi);

\end{tikzpicture}
\end{center}
lexer คือ ตัวที่ใช้ตัดคำ \\ 
parser คือ ตัวที่เอาไว้ check ไวย์ยากรณ์


\newpage
\section{context free grammar}
\begin{lstlisting}
program         -> initial? method*
initial         -> statement_list
method          -> label statement_list
statement_list  -> (statement)*
statement       -> intruction comment?(<END OF LINE> | <END OF LINE>)
                 | EMPTY
EMPTY           -> <END OF LINE>
instruction     -> R-TYPE reg reg reg
                 | I-TYPE reg reg field
                 | J-type reg reg 
                 | O-TYPE 
                 | .fill field
reg             -> INTEGER
field           -> offsetfield | label
label           -> WORD
comment         -> (WOED | INTEGER)*
factor          -> (+,-)factor | INTEGER
offsetfield     -> factor
\end{lstlisting}

\newpage
\section{ตารางการทำงาน}

\begin{tabular}{ |p{2cm}|p{6cm}|p{4cm}|  }
\hline
วัน/เดือน/ปี & รายละเอียดงาน &ผู้รับผิดชอบ \\
\hline
7/11/2562 & นัดคุยงาน วางแผน และแบ่งหน้าที่ในการทำงาน  & ทุกคน \\
\hline
11/11/2562 & นัดคุยรายละเอียดเกี่ยวกับ instruction  & วสุธันย์(600610773) \\ &&ศศิรัตน์(600610778) \\ &&เอกวิทย์(600610804) \\
          & คุยรายละเอียดเกี่ยวกับการทำ simulator & อาภัสสรา(600610802)  \\ && อานันท์(600610801) \\
\hline
11/11/2562 - 16/11/2562 &ทำงานของเองที่ได้รับมอบหมาย & ทุกคน\\
\hline
18/11/2562 &ทำส่วนของ... ให้เสร็จและนัดหมายคุยเพื่อปรับปรุงแก้ไขงานส่วนที่ทำ & อาภัสสรา(600610802)        อานันท์(600610801)\\
\hline
\end{tabular}
\subsection{หน้าที่การทำงานของสมาชิกในกลุ่ม}
วสุธันย์ กิติจีราพัฒน์ (600610773) ..... 

\end{document}
