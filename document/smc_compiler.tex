\documentclass[14pt]{article}
\usepackage[english,thai]{babel}
\usepackage[utf8x]{inputenc}
\usepackage{fonts-tlwg}
\usepackage{tikz}
\usetikzlibrary{shapes,arrows,snakes,positioning,trees}
\usepackage{tabularx}
\usepackage{multirow}
\usepackage[a4paper]{geometry}
\usepackage{hyperref}


\usepackage{listings}

\lstset{
  basicstyle=\itshape,
  xleftmargin=3em,
  literate={->}{$\rightarrow$}{2}
           {α}{$\alpha$}{1}
           {δ}{$\delta$}{1}
}


\begin{document}

\begin{titlepage}
    \vspace*{\stretch{1.0}}
    \begin{center}
       \Large\textbf{Computer Architecture}\\
       \textbf{Project 1}\\
       \vspace{2cm}
       \itshapeวสุธันย์ กิติจีราพัฒน์ (600610773) \\ ศศิรัตน์ มณีเจียร (600610778) \\ 

    \end{center}
    \vspace*{\stretch{2.0}}
 \end{titlepage}
\section{Infrastructure}
ในการทำงานของโปรแกรมจะแบ่งเป็น 2 ส่วนหลักๆ ก็คือ assembler และ simulator 
\begin{center}
    \tikzstyle{block} = [rectangle, draw, 
        text width=6em, text centered, rounded corners, minimum height=3em]
    \tikzstyle{line} = [draw, -latex']
    
    \begin{tikzpicture}[node distance = 3cm, auto]
        % Place nodes
        \node[draw=none,fill=none] (file) {files};
        \node [block, right of=file] (assembler) {assembler};
        \node [block, right of=assembler] (simulator) {simulator};
        \node[draw=none,fill=none,right of=simulator] (result) {result};
        
        % Draw edges
        \path [line] (file) -- (assembler);
        \path [line] (assembler) --  (simulator);
        \path [line] (simulator) --  (result);
    \end{tikzpicture}
    \end{center}
\section{Assembler}
แนวคิดในการทำงานในส่วนของ Compiler
\begin{center}
\tikzstyle{block} = [rectangle, draw, 
    text width=6em, text centered, rounded corners, minimum height=3em]
\tikzstyle{line} = [draw, -latex']

\begin{tikzpicture}[node distance = 2cm, auto]
    % Place nodes
    \node[draw=none,fill=none] (source) {Source code};
    \node [block, below of=source] (init) {lexer};
    \node [block, below of=init] (parser) {parser};
    \node [block, below of=parser] (semantic) {semantic analyze};
    \node [block, below of=semantic] (interpreter) {interpreter};
    \node[draw=none,fill=none,below of=interpreter] (bi) {binary};
    
    % Draw edges
    \path [line] (source) -- (init);
    \path [line] (init) -- node {token} (parser);
    \path [line] (parser) -- node {parsed tree} (semantic);
    \path [line] (semantic) -- node {program} (interpreter);
    \path [line] (interpreter) -- (bi);

\end{tikzpicture}
\end{center}
\parเริ่มต้นจากนำไฟล์ source code มา
lexer คือ ตัวที่ใช้ตัดคำ เพื่อนำ token ไปส่งให้ parser , parser คือ ตัวที่เอาไว้ check ไวย์ยากรณ์ ว่าในหนึ่งบรรทัด มีรูปแบบที่ถูกต้องหรือไม่ ต่อจากนั้น parser จะทำการสร้าง pared tree เพื่อส่งต่อ ไปให้ semantic analyze ซึ่ง semantic analyze ทำหน้าที่ในการเช็คข้อมูลในหนึ่งบรรทัดในเชิงความหมายของ instruction หรือ label เช่น ไม่ควรมี label ซ้ำกัน ต่อจากนั้น semantic analyze จะส่งข้อมูลที่ได้ส่งไปให้ interpreter เพื่อทำการแปลงจาก assembly เป็น binary



\newpage
\section{Instruction table}


\newcolumntype{L}[1]{>{\raggedright\arraybackslash}p{#1}}
\newcolumntype{C}[1]{>{\centering\arraybackslash}p{#1}}
\newcolumntype{R}[1]{>{\raggedleft\arraybackslash}p{#1}}

\begin{tabular}{C{1cm}|C{1cm}|C{1cm}|C{1cm}|C{1cm}|C{1cm}|C{1cm}|C{1cm}|C{1cm}}
\multicolumn{1}{L{0.7cm}}{} & \multicolumn{1}{l}{31} &\multicolumn{1}{r}{25}& \multicolumn{1}{l}{24 \hfil   22} &\multicolumn{1}{l}{21  \hfil19}& \multicolumn{1}{l}{18 \hfil \hfil 16} &\multicolumn{1}{l}{15}& \multicolumn{1}{r}{3}& \multicolumn{1}{l}{2 \hfil \hfil 0}\\
\hline
R-type & \multicolumn{2}{C{3cm}|}{x} & opcode & rs & rt & \multicolumn{2}{C{3cm}|}{x} & rd\\
\hline
I-type & \multicolumn{2}{C{3cm}|}{x} & opcode & rs & rt & \multicolumn{2}{R{3cm}}{offsetfield} \\
\hline
J-type & \multicolumn{2}{C{3cm}|}{x} & opcode & rs & rt & \multicolumn{2}{C{3cm}}{x} \\
\hline
O-type & \multicolumn{2}{C{3cm}|}{x} & opcode & \multicolumn{4}{R{3cm}}{x}\\
\hline

\end{tabular}

\section{context free grammar}
\paragraph{  }แนวคิดที่ได้จากการค้นคว้า ก็ได้แนวคิดของการทำ context free grammar โดยแนวคิดนี้จะเริ่มจากจุดใหญ่ๆก่อนแล้วค่อยย่อยมาเป็นจุดเล็กๆ ทำให้การทำงานง่ายขึ้น จาก context free grammar ก็ได้แนวคิดอีกเรื่องหนึี่งที่ต่อยอดมาจาก context free grammar ก็คือ abstact syntex tree คือเรื่องที่นำ context free grammar มาทำให้ดูง่ายขึ้นกว่าเดิม และจากการศึกษา ก็ได้นำมาเป็นแนวคิดในการเขียน code \par
แนวคิดของ context free grammar คือ เริ่มต้นจาก Assembly หนึ่งไฟล์จะประกอบไปด้วย initial หรือ method โดย initial จะเริ่มเป็น statement\_list แต่ method จะมี label ขึ้นก่อนแล้วตามด้วย statement\_list โดยใน statememt\_list จะมี statememt ซึ่งเก็บ instruction กับ commemt ซึ่งจะมีวิธีการเขียนที่เป็นไปได้ดังด้านล่างนี้ โดย instruction ก็จะแบ่งไปตาม type ต่างๆ มีรูปแบบที่ต่างกันไป 
\begin{lstlisting}
program         -> initial? method*
initial         -> statement_list
method          -> label statement_list
statement_list  -> (statement)*
statement       -> intruction comment?(<END OF LINE> | <END OF LINE>)
                 | EMPTY
EMPTY           -> <END OF LINE>
instruction     -> R-TYPE reg reg reg
                 | I-TYPE reg reg field
                 | J-TYPE reg reg 
                 | O-TYPE 
                 | .fill field
reg             -> INTEGER
field           -> offsetfield | label
label           -> WORD
comment         -> (WOED | INTEGER)*
factor          -> (+,-)factor | INTEGER
offsetfield     -> factor
\end{lstlisting}
ซึ่งหลักการของ comtext free grammar จะนำไปแปลงเป็น AST ซึ่งอยู่ในขั้นตอนของ parser ที่จะส่ง parsed tree ไปให้ semantic analyze ไปทำงานต่อไป

\newpage
\section{Abstract Syntex Tree}
\parจาก context free grammar จะนำมาแปลงให้เป็น Abstract syntex tree (AST) การทำ AST เพื่อให้การทำงานง่ายขึ้นกว่าเดิม โดยจะแยกว่า อันไหนเป็น initial หรือ method ที่ได้มาจาก parser  โดยมีตัวอย่างดังด้านล่าง \\
\tikzstyle{level 1}=[level distance=30mm, sibling distance=50mm]
\tikzstyle{level 2}=[level distance=40mm, sibling distance=27mm]
\tikzstyle{level 3}=[level distance=20mm, sibling distance=15mm]
\tikzstyle{level 4}=[level distance=20mm, sibling distance=5mm]
\begin{tikzpicture}[grow=right,->,>=angle 60]
%\begin{scope}[yshift=0]
  \node {paragraph tree}
    child [sibling distance=160mm]{node {methods}
    child {node {stAddr}
        child {node{.fill}
            child[-]{node{start}}}  
      }
      child {node {neg1}
        child {node{.fill}
            child[-]{node{-1}}}  
      }
      child {node {five}
        child {node{.fill}
            child[-]{node{5}}}  
      }
      child {node {done}
        child[-] {node{halt}}  
      }
      child {node{start}
        child{node{noop}}
        child{node{beq} 
            child[-] {node{start}}
            child[-] {node{0}} 
            child[-] {node{0}}  
      }
        child{node{beq} 
            child[-] {node{2}}
            child[-] {node{1}} 
            child[-] {node{0}}  
      }
        child{node{add} 
            child[-] {node{1}}
            child[-] {node{2}} 
            child[-] {node{1}}  
      }
      }
    }
    child {node {initial}
      child {node{lw}
        child[- ,sibling distance=7mm] {node{1}}
        child[- ,sibling distance=7mm] {node{2}} 
        child[- ,sibling distance=7mm] {node{1}}   
      }
      child {node{lw}
        child[- ,sibling distance=7mm] {node{five}}
        child[- ,sibling distance=7mm] {node{1}} 
        child[- ,sibling distance=7mm] {node{0}}   
      }
    };
\end{tikzpicture}




\newpage
\section{Simulator}
%เขียนตรงนี้ได้เลย
  

\newpage
\section{Assembly}
\subsection{Multiplication}
\begin{lstlisting}
        lw    0    2    mplier    $2 = 10383
        lw    0    3    mcand     $3 = 32766
        lw    0    4    pos1      $4 = 1
        lw    0    5    pos1      $5 = 1
        lw    0    6    pos16     $6 = 16
loop    beq   5    6              end goto end program when $5==16
        add   0    0    7         $7 = 0
        nand  4    2    7         $4 nand $2 -> $7
        nand  7    7    7         nand for check bit
        beq   7    0              skip goto skip program when $7==0
        add   3    1    1         $1 = $3 + $1
skip    add   3    3    3         shift the multiplicand register left 1 bit
        add   0    0    7         $7 = 0
        lw    0    7    pos1      $7 = 1
        add   5    7    5         $5 = $5 +1
        add   4    4    4         shift the multiplier register left 1 bit
        beq   0    0              loop go back to the begining of the loop
end     halt                      end of program
mcand   .fill 32766
mplier  .fill 10383
pos16   .fill 16
pos1    .fill 1
\end{lstlisting}

\subsection{Combination}
\begin{lstlisting}
       lw    0   1   n      $1 == n 
       lw    0   2   r      $2 == r
       lw    0   6   func   prepare to call combi. $2=combi 
start  jalr  6   7          call combi: $7=return address
       lw    0   1   n      $1 == n
       lw    0   2   r      $2 == r
       halt
combi  beq   0   2   exit   go to exit when $2 == 0
       beq   1   2   exit   go to exit when $2 == $1
       lw    0   4   pos1   $4 == 1
       sw    5   7   stack  save return address on stack
       add   5   4   5      increment stack pointer
       sw    5   1   stack  save $1 on stack
       add   5   4   5      increment stack pointer
       sw    5   2   stack  save $2 on stack
       add   5   4   5      increment stack pointer
       lw    0   4   neg1   $4 == -1
       add   1   4   1      $1 = $4+$1
       jalr  6   7          call combi: $7=return address
       lw    0   4   neg1   $4 == -1
       add   5   4   5      decrement stack pointer
       lw    5   2   stack  recover $2
       add   5   4   5      decrement stack pointer
       lw    5   1   stack  recover $1
       add   1   4   1      $1 = $4+$1
       add   2   4   2      $2 = $4+$2
       jalr  6   7          call addr $7
       lw    0   4   neg1   $4 == -1
       add   5   4   5      decrement stack pointer
       lw    5   7   stack  recover return address on stack
       jalr  7   0          call addr $7
exit   lw    0   4   pos1   return 1
       add   3   4   3      result combi
       jalr  7   0          call addr $7
func   .fill combi
pos1   .fill 1
neg1   .fill -1
n      .fill 7
r      .fill 3
stack  .fill 0
    
\end{lstlisting}

\subsection{Fibonacci}
\begin{lstlisting}
       lw    0    1    n        $1 == n     
       lw    0    6    func     prepare to call fibo : $2=fibo 
start  jalr  6    7    call     fibo: $7=return address
       halt
fibo   beq   1    0    done     go to done when $1 == 0
       lw    0    4    pos1     $4 == 1
       beq   1    4    done     go to done when $1 == 1
       sw    5    7    stack    save return address on stack
       add   5    4    5        increment stack pointer
       sw    5    1    stack    save $1 on stack
       add   5    4    5        increment stack pointer
       lw    0    4    neg1     $4 == -1
       add   1    4    1        $1 = $4+$1
       jalr  6    7             call fibo: $7=return address
loop1  lw    0    4    neg1     $4 == -1
       add   5    4    5        decrement stack pointer
       lw    5    1    stack    recover $1
       add   4    4    4        $4 == -2
       add   1    4    1        $1 = $4+$1
       jalr  6    7             call fibo: $7=return address
loop2  lw    0    4    neg1     $4 == -1
       add   5    4    5        decrement stack pointer
       lw    5    7    stack    recover return address on stack
       jalr  7    0             call addr $7
done   add   0    1    2        $1 == $2
       add   2    3    3        $3 = $3 + $2
       jalr  7    0             call addr $7
func   .fill fibo
n      .fill 5
pos1   .fill 1
neg1   .fill -1
stack  .fill 0   
\end{lstlisting}

\newpage
\vspace*{\fill}
\noindent
\makebox[\textwidth]{\Huge ภาคผนวก}

\vfill

\newpage
\section{ตารางการทำงาน}

\begin{tabular}{ |p{2cm}|p{6cm}|p{3cm}|p{2cm}|  }
\hline
วัน/เดือน/ปี & รายละเอียดงาน &ผู้รับผิดชอบ & จำนวนชั่วโมง\\
\hline
7/11/2562 & นัดคุยงาน วางแผน และแบ่งหน้าที่ในการทำงาน  & ทุกคน & 2 \\
\hline
11/11/2562 & นัดคุยรายละเอียดเกี่ยวกับ instruction  & วสุธันย์(600610773) \\ &&ศศิรัตน์(600610778) \\ &&เอกวิทย์(600610804) \\
          & คุยรายละเอียดเกี่ยวกับการทำ simulator & อาภัสสรา(600610802)  \\ && อานันท์(600610801) \\
\hline
11/11/2562 - 16/11/2562 &ทำงานของเองที่ได้รับมอบหมาย & ทุกคน\\
\hline
18/11/2562 &ทำส่วนของ Asembly ให้เสร็จและนัดหมายคุยเพื่อปรับปรุงแก้ไขงานส่วนที่ทำ & อาภัสสรา(600610802)        อานันท์(600610801)\\
\hline
\end{tabular}
\subsection{หน้าที่การทำงานของสมาชิกในกลุ่ม}
วสุธันย์ กิติจีราพัฒน์ (600610773) ..... 
\section{Reference}
\begin{enumerate}
\item\url{https://www.cs.rochester.edu/~nelson/courses/csc_173/grammars/cfg.html}
\item\url{https://medium.com/basecs/leveling-up-ones-parsing-game-with-asts-d7a6fc2400ff}
\item\url{http://www1.cs.columbia.edu/~sedwards/classes/2003/w4115f/ast.9up.pdf}
\item\url{http://matt.might.net/articles/grammars-bnf-ebnf/}
\item\url{https://ruslanspivak.com/lsbasi-part1/}
\item\url{https://ruslanspivak.com/lsbasi-part2/}
\item\url{https://ruslanspivak.com/lsbasi-part3/}
\item\url{https://ruslanspivak.com/lsbasi-part4/}
\item\url{https://ruslanspivak.com/lsbasi-part5/}
\item\url{https://ruslanspivak.com/lsbasi-part6/}
\item\url{https://ruslanspivak.com/lsbasi-part7/}
\item\url{https://ruslanspivak.com/lsbasi-part8/}
\item\url{https://ruslanspivak.com/lsbasi-part9/}
\item\url{https://ruslanspivak.com/lsbasi-part10/}
\item\url{https://ruslanspivak.com/lsbasi-part11/}
\item\url{https://ruslanspivak.com/lsbasi-part12/}
\item\url{https://ruslanspivak.com/lsbasi-part13/}
\item\url{https://ruslanspivak.com/lsbasi-part14/}
\item\url{https://ruslanspivak.com/lsbasi-part15/}
\item\url{https://ruslanspivak.com/lsbasi-part16/}
\item\url{https://ruslanspivak.com/lsbasi-part17/}
\end{enumerate}


\end{document}
